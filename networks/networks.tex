\documentclass{report}
\usepackage[utf8]{inputenc}
\usepackage{amsmath,amsthm,amsfonts,amssymb,amscd}
\usepackage[a4paper,hmargin=0.8in,bottom=1.3in]{geometry}
\usepackage{lastpage,enumerate,fancyhdr,mathrsfs,xcolor,graphicx,listings,hyperref,enumitem}
\makeatletter
\def\maxwidth#1{\ifdim\Gin@nat@width>#1 #1\else\Gin@nat@width\fi}
\makeatother
\newcommand{\protorow}[4]{#1 & #2 & #3 & #4 \\ \hline}
\newcommand{\mygraphic}[1]{
\begin{center}
    \includegraphics[width=\maxwidth{15cm}]{#1}
\end{center}
}
\author{Hardik Rajpal}
\title{Of packets and their journeys.}
\begin{document}
\maketitle
\tableofcontents
\chapter{Misc}
\section{Big Fat Protocol Table}
\begin{center}
\begin{tabular}{| p{3cm} | p{3cm} | p{4cm} | p{4cm} | }
\hline
\protorow{Name}{Operates at Layer}{Function}{Remarks}
\protorow{ARP}{}{Returns the MAC address corresponding to an IP address}{\hyperref[sec:arp]{Further discussion.}}
\end{tabular}
\end{center}
\chapter{Link Layer}
\section{ARP Protocol}
\label{sec:arp}
\begin{itemize}
\item Operates at link layer.
\item Used to find the MAC address m corresponding to an IP address a.
\item Broadcasts "who is a? tell srcIPAddress." message. Host with IP address a replies.
\begin{itemize}
\item Ex: \texttt{who is 10.11.63.71? tell 10.09.63.43.}
\end{itemize}
\item Each intermediate host maintains a cache of IP to MAC translations and updates its cache on parsing ARP replies and requests.
\item The requesting host saves the reply MAC in its cache.
\item Entries in said cache timeout periodically.
\item The packet format is:
\mygraphic{rsrc/arppacket.png}
\begin{itemize}
\item Hardware type specifies what link level technology we're using. For ex, it's set to 1 for ethernet.
\item Protocol Type refers to higher level protocol. It's 0x0800 for IP.
\item HLEN specifies length of the MAC address in bits.
\item PLEN specifies length of the protocol address in bits. It's 32 for IP address in bits.
\item Operation can be: request or reply.
\end{itemize}
\item The terms involved are:
\begin{itemize}
\item Originator: Host that generates ARP request.
\item Target: Host replying to the ARP request. It updates its cache with srcIP, srcMAC.
\end{itemize}
\item When a host has to forward a datagram that specifies a destination IP address (that is within the LAN),
\begin{enumerate}
    \item It first checks its ARP cache for a map from dstIP to MAC.
    \item If no entry is found, it broadcasts an ARP request.
    \item While the request and reply move through the LAN, intermediate hosts refresh their caches.
\end{enumerate}
\item Note: Intermediate hosts NEVER reply to ARP requests.
\end{itemize}
\subsection{Gratuitous ARP}
\begin{itemize}
\item Generated by a host to inform others of its IP and MAC address.
\item According to \href{https://networkengineering.stackexchange.com/questions/7713/how-does-gratuitous-arp-work}{this,} gratuitous ARPs are request packets and not reply packets.
\item Both IPdst and IPsrc are set to IP host, and src MAC is set to host MAC.
\item dst MAC is the broadcast address: \texttt{ff:ff:ff:ff:ff:ff}
\item No reply is expected.
\item Gratuitous ARP is used to:
\begin{enumerate}
\item Inform hosts of changes to my IP or MAC address.
\item Inform hosts that a host is now available.
\item Help rectify ARP entries.
\item Report IP address conflicts (duplicate IP addresses).
\item Inform learning bridges of the new location of the host, or the location of a new host.
\end{enumerate}
\item Note that since ARP is a stateless protocol, even replies that were never requested are parsed and processed and thus, can function as gratuitous ARPs.
\end{itemize}
\end{document}