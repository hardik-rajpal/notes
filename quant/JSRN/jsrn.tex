\documentclass{article}
\usepackage[utf8]{inputenc}
\usepackage{amsmath,amsthm,amsfonts,amssymb,amscd}
\usepackage[a4paper,hmargin=0.8in,bottom=1.3in]{geometry}
\usepackage{lastpage,enumerate,fancyhdr,mathrsfs,xcolor,graphicx,listings,hyperref,enumitem}
\title{Reading Notes from JS's G
uide for Job Applications}
\author{Hardik Rajpal}
\begin{document}
\maketitle
\section{Randomness}
\begin{itemize}
\item Financial markets involve a lot of uncertainty; predictions are difficult.
\item Probability provides a good framework for making decisions when uncertainty is involved.
\item One-off future events are more complicated to draw (even probabilistic) predictions about. Said predictions vary on asymmetry of information and personal experiences.
\item Probability can and is used to reason about \textbf{knowable unknowns}: events that have a deterministic value (ex. events in the past) but said value is not known to anyone (within reach). 
\item There are ``different types'' of randomness (random variables):
\begin{enumerate}
    \item Discrete
    \begin{itemize}
        \item Specially, Binary.
    \end{itemize}
    \item Continuous
\end{enumerate} 
\end{itemize}
\section{Counting}
\begin{itemize}
\item With a \textbf{finite number of outcomes}, each of which is \textbf{equally likely}, evaluating the probability of each event (set of outcomes) is equivalent to figuring out its cardinality.
\begin{center}
    p[criteria is fulfilled] = $\frac{\text{\# outcomes that fulfill criteria}}{\text{total number of possible outcomes}}$
\end{center}
\item Outcomes that are not equally likely can be viewed events consisting of atomic, equally likely outcomes to simplify calculations. 
\end{itemize}
\section{Probability, Independence, Random Variables, Expectation}
\begin{itemize}
\item Same old, same old.
\end{itemize}
\section{Confidence Intervals}
\begin{itemize}
\item There are a few ways to consider a range of values of a random variable for processing:
\begin{enumerate}
\item Study the literal min,max values of the RV.
\item Measure E[$(X-\mu)^2$] or E[$\|X-\mu\|$]
\item Describe a confidence interval: an interval containing most of the probability mass of the random variable. Its confidence is given by P(X$\in$ CI).
\end{enumerate}
\end{itemize}
\section{Conditional Probability}
\begin{itemize}
\item A \textbf{prior} refers to the initial probability function, before any information is considered.
\item On receiving information, we ``update'' our beliefs.
\end{itemize}
\section{Making Markets}
\begin{itemize}
\item To effect a trade, we need to specify the following:
\begin{enumerate}
\item The object being traded (often implied by the context).
\item The direction: buy or sell.
\item The price of the transaction.
\item The quantiy or size of the transaction.
\end{enumerate}
\item To buy, we indicate bids; to sell, we indicate offers (asks).
\item ``I'm 2 bid for 10 '' $\implies$ I want to buy 10 units for the price of \$2 each.
\item ``I have 10 at 4 '' $\implies$ I want to sell 10 units for the price of \$4 each.
\item ``I am 2 at 4, 10 up '' $\implies$ I want to sell 10 units for the price of \$4 each, buy 10 units at the price of \$2.
\item These phrases create orders (irl or on exchange platforms), which stay active and anyone can trade against them until its traded or you say ``I'm out.''
\item To buy someone's bid order, say ``sold.'' To sell to someone's ask order, say ``take' em.'' 
\item For partial fills, attach the quantity to the comment.
\item Once an order is filled, (the entire quantity has been traded), it cannot be traded against again.
\end{itemize}
\subsection{The Strategy}
\begin{itemize}
\item If you think E[X] is $\mu$, you should be willing to buy for less and to sell for more.
\item The factors to consider before hitting the red button:
\begin{itemize}
\item On average, do you stand to profit.
\item Can you lose a lot of money (WCS)? Losing money is bad because it keeps you from participating in great opportunities to trade in the future if your capital is depleted.
\item How much are you making? Bidding too low to maximize the profit in every outcome means there's a low chance of being sold the contract. Bidding right at the mean $\mu$ means the profit is limited.
\end{itemize}
\item We want to balance our likelihood of a trade and the expected profit.
\item Here's a market on the E[die roll]: ``3 at 4, 10 up.''. (Buy, bid price first, followed by ask price, followed by quantity.)
\end{itemize}
\end{document}