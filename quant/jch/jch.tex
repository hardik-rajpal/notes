\documentclass{article}
\usepackage[utf8]{inputenc}
\usepackage{amsmath,amsthm,amsfonts,amssymb,amscd}
\usepackage[a4paper,hmargin=0.8in,bottom=1.3in]{geometry}
\usepackage{lastpage,enumerate,fancyhdr,mathrsfs,xcolor,graphicx,listings,hyperref,enumitem}
\author{Hardik Rajpal}
\title{Notes from John C. Hull's famed textbook.}
\hbadness 100001
\begin{document}
\maketitle
\section{Chapter 1: Introduction}
A \textbf{derivative} can be defined as a financial instrument whose value depends on (or
derives from) the values of other, more basic, underlying variables. The variables 
underlying a derivative are often the prices of traded assets. A stock option is a
derivative whose price is dependent on the price of the stock. Note that derivatives
can be dependant on almost any variable: from the price of hogs to the amount
of snow in the Alpes.
\subsection{Exchange-Traded Markets}
A derivatives exchange is a market where individuals trade standardized contracts that
have been defined by the exchange. Traditionally, derivative exchanges used the 
\emph{open outcry system}; physically meeting up on the floor of the exchange, shouting
and using a set of complicated hand signals. This is fortunately being replaced by
electronic trading. The replacement has led to a growth in algorithmic trading,
also known as blackbox trading, automated trading, high-frequency trading, or robo trading,
or beep-boop-machine-make-money-out-of-thin-air trading. The last one is rarely used in
the industry.
\subsection{Over-the-counter Markets}
This is an alternative to the exchange market and has become \textbf{much larger} than it in terms
of the total volume of trading. Trades are done over the phone and usually between
two fininsts or between a fininst and one of its clients (typically a corporate treasurer
or a fund manager). Fininsts often act as \textbf{market makers} for the more commonly
traded instruments; they are prepared to quote a bid price (to buy) and offer price (to sell).
There is a credit risk associated with over-the-counter trade, while exchanges have
organized themselves to eliminate virtually all credit risk.
\subsection{Forward Contracts}

\end{document}