\documentclass{article}
\usepackage[utf8]{inputenc}
\usepackage{amsmath,amsthm,amsfonts,amssymb,amscd}
\usepackage[a4paper,hmargin=0.8in,bottom=1.3in]{geometry}
\usepackage{lastpage,enumerate,fancyhdr,mathrsfs,xcolor,graphicx,listings,hyperref,enumitem}
\author{Hardik Rajpal}
\title{Notes from John C. Hull's famed textbook.}
\hbadness 100001
\begin{document}
\maketitle
\section{Chapter 1: Introduction}
A \textbf{derivative} can be defined as a financial instrument whose value depends on (or
derives from) the values of other, more basic, underlying variables. The variables 
underlying a derivative are often the prices of traded assets. A stock option is a
derivative whose price is dependent on the price of the stock. Note that derivatives
can be dependant on almost any variable: from the price of hogs to the amount
of snow in the Alpes.
\subsection{Exchange-Traded Markets}
A derivatives exchange is a market where individuals trade standardized contracts that
have been defined by the exchange. Traditionally, derivative exchanges used the 
\emph{open outcry system}; physically meeting up on the floor of the exchange, shouting
and using a set of complicated hand signals. This is fortunately being replaced by
electronic trading. The replacement has led to a growth in algorithmic trading,
also known as blackbox trading, automated trading, high-frequency trading, or robo trading,
or beep-boop-machine-make-money-out-of-thin-air trading. The last one is rarely used in
the industry.
\subsection{Over-the-counter Markets}
This is an alternative to the exchange market and has become \textbf{much larger} than it in terms
of the total volume of trading. Trades are done over the phone and usually between
two fininsts or between a fininst and one of its clients (typically a corporate treasurer
or a fund manager). Fininsts often act as \textbf{market makers} for the more commonly
traded instruments; they are prepared to quote a bid price (to buy) and offer price (to sell).
There is a credit risk associated with over-the-counter trade, while exchanges have
organized themselves to eliminate virtually all credit risk.
\subsection{Forward Contracts}
A type of derivative that is an agreement to buy or sell an asset at a certain
future time for a certain price (is a \textbf{forward contract}). In contrast,
a \textbf{spot contract} is an agreement to buy or sell an asset today. These
appear in over-the-counter trades. Each party
involved in a forward contract can assume one of two positions:
\begin{enumerate}
    \item the \textbf{long} position: agreeing to \textbf{buy} the asset at the specified date and price.
    \item the \textbf{short} position: agreeing to \textbf{sell} the asset at the specified date and price.
\end{enumerate}
The payoff from a \textbf{long} position in a forward contract on one unit of an asset is: $S_T - K$, where
$K$ is the delivery price (the one agreed to by both parties) and $S_T$ is the spot price of the asset
at maturity of the contract. The payoff in the \textbf{short} position is negative of that in the long
position: $K-S_T$. As forward contracts cost nothing to enter, these payoffs are also the net gain or loss
associated with the contract.\\
Forward contracts can be used in the landscape of stocks to benefit from expected changes in stock prices:
\begin{itemize}
    \item If a stock is expected to drop ($\implies$ have a lower $S_T$ in the near future), enter a forward
    contract to buy the stock in the future at $K<\text{current price}$, having sold it now at the current price.
    \item If a stock is expected to rise ($implies$ have a higher $S_T$ in the near future), enter a forward
    contract to sell the stock in the future at $K>\text{current price}$, having sold it now at the current price.
\end{itemize}
Both these points assume enough profit is made to justify any loans involved.
\subsection{Futures Contracts}
They are also agreements to buy/sell an asset at a certain time at a certain price, but are traded on an exchange.
The commodities include pork bellies, live cattle, sugar, wool, lumber,
copper, aluminum, gold, and tin. The financial assets include stock indices, currencies,
and Treasury bonds. Future prices are regularly reported in the financial press.
\subsection{Options}
Options are traded both on exchanges and in the over-the-counter market. There are two types:
\begin{enumerate}
    \item Call option: Gives the holder the right to buy the underlying asset by a certain date for a certain price.
    \item Put option: Gives the holder the right to sell the underlying asset by a certain date for a certain price.
\end{enumerate}
The specified price is called the \textbf{exercise price or strike price}, while
the specified date is called the \textbf{expiration date or maturity}. Note that 
\textbf{American} options can be exercised at any date on or before the expiration, while
\textbf{European} options can only be exercised on the expiration date.\\
The option gives the holder the right to exercise a trade; they don't have to actually make
the trade. This freedom comes at a cost; precisely, the cost to acquiring an option.
\subsubsection{Properties of Options}
\begin{itemize}
    \item The price of a call option decreases as the strike price increases,
    while the price of a put option increases as the
    strike price increases. (Is this because the prices is in general expected to drop,
    and hence buying at a higher price (call option with high strike price)
    is less preferred while selling at a higher price (put option with higher strike price)
    is more preferred?).
    \item However, both types of options become more valuable as 
    their time to maturity increases.
\end{itemize}
Consider a case where an investor buys a call option with the asset as 100 shares of Google,
the strike price as \$520 per share and the expiration date as 18/12/10, at a price of \$32 per share.
\begin{itemize}
    \item The investor has spent \$3200 in acquiring the option.
    \item If Google's share price doesn't rise above \$520 by the expiration date,
    the investor has no reason to exercise the option (paying \$520 for a share while the
    bid price is much below that) and has thus lost \$3200.
    \item If before the expiration date, Google's share price rises above \$520, the investor can
    exercise the option; buy the shares at \$520, and sell them at the current price $S_T$, making
    a profit of $100(S_T-(520+32))$.
    \item Note that we're neglecting TVoM here.  
\end{itemize}
The person who sells the put option is obliged to do the buying at the option's maturity (or before).
Consider a case where an investor sells a put option with on 100 shares of Google with the 
strike price as \$480 per share and the expiration date as 18/09/10, at a price of \$22.2 per share.
\begin{itemize}
    \item Immediately, the investor receives a cash inflow of $100\times22.2=\$2220$
    \item Suppose Google's share price up to the expiration date remains above \$480, the buyer of the
    option has no incentive to sell his shares at \$480, and doesn't exercise the option.
    \item Suppose Google's share price $S_T$ is below \$480 before the expiration date, the buyer of the option
    sells the shares back to the investor at \$480, and buys them at $S_T$ making a profit of: $100\times(480-(S_T+22.2))$.
    \item In this case, the investor is forced to buy shares worth $S_T<480$ at 480, and suffers a net loss of:
    $100\times(480-(S_T+22.2))$.
    \item The investor's gain can be defined as 100 times $S_T+22.2 - 480$ when $S_T < 480$ and as 22.2 when $S_T>480$.
\end{itemize}
The diagrams below summarize the two analyses with the assumption
of European options.
\begin{center}
    \includegraphics[width=6in]{rsrc/trading-options.png}
\end{center}
Note that there are four types of participants in the options market with
the choice of put/call option and buying or selling the option. Buyers are said
to take \textbf{long positions} and sellers are said to take \textbf{short positions}.
Selling an option is also known as \textbf{writing an option}.
\subsection{Types of Traders}
Derivatives markets have been successful mainly because of the variety of traders
they have attracted and their great deal of liquidity. When an investor wants to take
one side of a contract, there is usually no problem in finding someone who is prepared
to take the other side. Traders can be broadly divided into three categories:
\begin{enumerate}
    \item Hedgers: use derivatives to reduce the risk they face from potential future movements
    in a market variable.
    \item Speculators: use derivatives to bet on the future direction of a market variable.
    \item Arbitrageurs: take offsetting positions in two or more instruments to lock in a profit. 
\end{enumerate}
\subsection{Hedgers}
Hedgers can use future contracts to, for example, hedge the risk of foreign exchange. Companies
that agree to pay a fixed amount in a currency, say C2, different from their operating currency,
say C1, can buy futures of the amount in C2 at a fixed price in C1; hence avoiding any loss
due to fluctuations in the exchange rate between C1 and C2. Conversely, a company set to receive
a payment in C2 in the future can sell futures of C1 at the current exchange rate to hedge
the foreign exchange risk. It should be noted that the companies may have been better off
without using the future contracts, but they could have been worse off too.\\
``The purpose of hedging is to reduce risk. There is no guarantee that the outcome with
hedging will be better than the one without hedging.''\\
\subsubsection{Hedging with Options}
We use options to hedge against the possible decline in the price of shares held by us. One method
is to buy put options at a high enough strike price $K$ to lower-bound our possible loss by $CP-K$,
where $CP$ is the current price of the shares.
\begin{itemize}
    \item If the price drops below K, we exercise our option
    to sell the shares at $\$K$ and buy them back at the lower price.
    \item If the price doesn't drop below K, we don't exercise our option, we have lost the amount
    we paid for the option, but we've been protected from the risk.
\end{itemize}
We note that options provide a lower bound for the loss incurred by price dips, while retaining
the possibilty to earn from price rises. Futures, however, fix the price to protect against
price dips but also keep us from benefitting from price rises.
\subsection{Speculators}
Speculators are either betting on the price of an asset going down or going up.
\subsubsection{With Futures}
Suppose a speculator believes that C2 will be increasingly more valuable than C1
over the next few months. The speculator can:
\begin{itemize}
    \item Purchase, say 250,000 of C2 right away and sell it when it becomes more valuable.
    \item Enter the long position on a future implying the purchase of 250,000 of C2 at a rate $K$
    lower than the expected exchange rate in the future. Then, on the expiration date, we 
    purchase 250,000 of C2 at this lowered rate and immediately sell it for a profit given by:
    $250,000\times(S_T-K)$
\end{itemize}
Note that if the futures's promised price is lower than the current exchange rate, the profit obtained
is even larger than that obtained by the first option, while the loss is reduced in magnitude too.
The futures market allows the speculator to take a large speculative position with a relatively
small initial outlay. 
\subsubsection{With Options}
Options allow for more profit (and equivalently, more loss, albeit capped at
the capital) to be made with the same investment capital,
than would be made by purchasing shares rightaway. This is because options themselves are much cheaper
than shares. The only reason the profit/loss is magnified is because you can buy more options with the
same capital right now than you can buy shares. Summary: Good outcomes are magnified, while with
bad outcomes, the entire initial investment is lost.
\subsubsection{Comparison}
It's important to note that with futures, both the loss and gain are unbounded but with options, the 
loss is bounded below by the amount paid for the options.
\subsection{Arbitrageurs}

\end{document}